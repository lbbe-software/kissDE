\documentclass[a4paper,10pt]{article}
\usepackage[utf8]{inputenc}

%opening
\title{}
\author{}

\begin{document}

\maketitle

\section{kissDE main steps}
\begin{itemize}
 \item Les comptages sont normalisés avec \textbf{DESeq} (utile dans le cas où on observe globalement une plus forte exception dans une condition que dans les autres)
 \item filtre sur les comptages : si les comptages globaux pour le variant du haut et pour le variant du bas sont trop bas, ils ne sont pas testés
 \item tests sur les comptages normalisés
 \item pvalues ajustées pour les tests multiples
 \item calcul du delta PSI
 \item flag low counts :\\
si dans au moins $n-1$ conditions ($n$ nombre de conditions $\geq 2$) un événement a des comptages faibles il est signalé
\end{itemize}

\section{Retour sur le calcul du delta PSI (delta f si SNP)}
Correction des comptages :
\begin{itemize}
 \item si on a des données en sortie de kissplice avec l'option \texttt{--counts} par défaut (\texttt{--counts 0} qui donne \texttt{...|C1\_25|C2\_1|C3\_14.....} dans les headers) :\\
  $\Rightarrow$ on n'a pas d'info sur les comptages sur les jonctions
 \begin{enumerate}
 \item on calcule un facteur correctif qui prend en compte les longueurs apparentes des reads:\\
   $correctFactor\:=\: \frac{length\_upper\_path + read\_length - 2 \times overlap + 1}{length\_lower\_path + read\_length - 2\times overlap + 1}$ avec :
 \begin{itemize}
  \item \texttt{upper\_path} est toujours l'isoforme d'inclusion (+long)/\texttt{lower\_path} est toujours l'isoforme d'exclusion
  \item \texttt{overlap} : paramètre dans kissreads, overlap min du read sur la bubble, par défaut \texttt{k+1} (pour couvrir au moins une jonction)
   \end{itemize}
            \item on corrige les comptages du chemin du haut (le plus long) car c'est celui sur lequel on est susceptible de voir plus de reads :\\
            $comptages\_upper\_path \:=\:  \frac{comptages\_upper\_path}{correctFactor} $
 \end{enumerate}
 \item si on a des données en sortie de kissplice avec l'option \texttt{--counts} à \texttt{2} (qui donne \texttt{...|AS1\_12|AB1\_7|S1\_0|ASSB1\_1 ou |AB1\_13|.....} dans les headers) :\\
 $\Rightarrow$ on a des infos sur les comptages sur les jonctions : sur le chemin du haut, le nombre de reads est \texttt{AS+AB-ASSB} (les \texttt{ASSB} sont comptés deux fois, dans \texttt{AS} ou \texttt{AB} et dans \texttt{ASSB})
  \item si on n'a pas de reads qui couvrent les deux jonctions, on va simplement diviser par 2 le nombre de reads vu sur le chemin du haut (globalement les comptages seront deux fois plus importants car on a deux jonctions sur le chemin du haut contre une seule sur le chemin du bas), 
  \item si on a des reads sur la jonction ASSB on veut diviser par un peu moins que 2 :
  \begin{enumerate}
%    \item \texttt{comptages\_upper\_path <- AS+AB-ASSB }
    \item avant correction : $comptages\_upper\_path \:=\: AS+AB-ASSB $
    \item après : $comptages\_upper\_path\:=\:\frac{comptages\_upper\_path}{2-\frac{ASSB}{AS+SB-ASSB}}$
  \end{enumerate}
 \end{itemize}
 
 Calcul du delta PSI :
 \begin{itemize}
\item calcul des psi individuels par réplicats :\\
après correction des comptages, si les comptages pour le chemin du haut et pour le chemin du bas sont trop faibles, on ne calcule pas le psi individuel
\item psi moyens par conditions :\\
 si plus de la moitié des psi individuels qui servent à calculer le psi moyen n'ont pas été calculés à l'étape précédente, on ne sort pas le psi moyen
\item delta psi :\\
non sorti si un des psi moyen n'a pas pu être estimé
\end{itemize}





\end{document}
